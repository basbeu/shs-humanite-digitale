\documentclass[french,a4paper]{article}

\usepackage[utf8]{inputenc}
\usepackage[T1]{fontenc}
\usepackage{babel}
\usepackage[margin=2.5cm]{geometry}
\usepackage{enumitem}

\PassOptionsToPackage{hyphens}{url}\usepackage{hyperref}

\setlength{\parindent}{2em}          % Espace horizontal au début des paragraphes
\setlength{\parskip}{0.5em}          % Espace vertical entre les paragraphes
\renewcommand{\baselinestretch}{1.1} % Espace vertical entre les lignes

\title{La Première Guerre mondiale à travers les tensions entre les Suisses romands et alémaniques}
\author{\bsc{Beuchat} Bastien \and \bsc{Ding} Markus \and \bsc{Jollès} Eric \and \bsc{Mamie} Robin}
\date{20 novembre 2019} % Date du premier rendu intermédiaire
%\date{4 décembre 2019} % Date du second rendu intermédiaire

%\date{25 mars 2020} % Date du premier rendu intermédiaire avec méthodologies et interprétations
%\date{22 avril 2020} % Date du second rendu intermédiaire avec méthodologies et interprétations
%\date{6 mai 2020} % Date du rendu final

\begin{document}

\maketitle

\section*{Introduction}

\subsection*{Contexte historique}

En 1914, l'Europe est sous haute tension.
L'assassinat de François Ferdinand est l'étincelle qui déclenche la Première Guerre mondiale.
La Suisse se retrouve alors au centre du conflit, encerclée de toutes parts par les nations belligérantes qui constituent la Triple-Entente et la Triple-Alliance.
De part leurs proximités culturelles et linguistiques avec ces belligérants, la Suisse romande, la Suisse alémanique et le Tessin ont différentes visions du conflit \cite{division}.

Suite à son élection controversée face à Theophil Sprecher \cite{wahl} le 3 août 1914, \textbf{Ulrich Wille} occupe le poste de général de l'armée suisse.
Né à Hambourg en Allemagne, parlant \textit{Hochdeutsch} à la place d'un dialecte suisse allemand, sa sympathie ouverte envers l'Empire germanique et ses décisions durant la guerre -- notamment la proposition d'entrer en guerre au côté des Empires centraux \cite{krieg} -- font de lui un personnage très polarisant \cite{delire}.

À la fin de l'année 1915, \textbf{l'affaire des colonels} éclate au sein de l'état-major général de l'armée suisse.
En effet, le bulletin de l'état-major et des dépêches diplomatiques décryptées furent transmis par les colonels Friedrich Moritz von Wattenwyl et Karl Egli à l'Allemagne et à l'Autriche.
André Langié en fut le lanceur d'alerte \cite{whistle_blower}.
La non condamnation des deux accusés attise les tensions entre les Suisses romands et allemands, notamment dans la presse francophone \cite{verdict}.

Des appels à l'apaisement sont lancés par la Confédération \cite{apaisement}, mais aussi par certaines figures importantes de la période comme \textbf{Carl Spitteler} lors de son discours \og \textit{Unser Schweizer Standpunkt} \fg{} (\og \textit{Notre Point de vue suisse} \fg{}) \cite{standpunkt} \cite{rede_spitteler} .
La plupart des journaux des deux régions linguistiques reprenant la propagande française ou allemande \cite{propagande} \cite{place} n'aident pas à l'apaisement général. \textbf{L'invasion de la Belgique}, et plus précisément le feu de la bibliothèque universitaire de Louvain, est un sujet divisant l'opinion suisse.
D'un côté, l'agression et la férocité allemandes sont dénoncées \cite{massacre}, alors que de l'autre, l'indifférence prévaut.

En outre, le conseiller fédéral \textbf{Arthur Hoffmann} et Robert Grimm ont tenté de négocier une paix séparée entre la Russie et l'Allemagne.
La découverte de cette intervention secrète crée de grands remous en Romandie ainsi que des pressions de la part de l'Entente qui craignait un retrait russe.
Les actions de Hoffmann étaient contraires aux principes de neutralité et il a été forcé à démissionner.
Son remplaçant est le socialiste genevois Gustave Ador, plutôt favorable à l'Entente.
Son élection permet de calmer autant les tensions internes qu'externes.

En février 1918, un comité est formé à Olten afin de coordonner une \textbf{grève générale} dans tout le pays.
Composé uniquement de Suisses allemands, les Romands sont suspicieux de ce mouvement.
Les Romands l'accusent de saboter la joie de la victoire de l'Entente en déclarant la grève le jour de la fin de la guerre.
La grève est moins suivie et respectée en Suisse romande. \cite{sprachenfrieden}

\section*{Corpus et références bibliographiques}

En tant que source primaire pour notre analyse, nous utilisons les archives de la presse suisse.
Nous y accédons à travers la plate-forme Impresso.
Notre travail a pour but de comparer les divergences de points de vue entre Suisses romands et allemands durant la Première Guerre mondiale.
Dans cette optique, nous travaillons avec des archives des presses romande et alémanique.
Nous étudions donc les articles parus dans la Gazette de Lausanne et le Journal de Genève pour analyser le point de vue francophone et ceux de la Neue Zürcher Zeitung (NZZ) pour l'analyse germanophone.
Nous travaillons principalement sur des articles parus pendant les moments critiques décrits dans la première partie.

L'interface Impresso permettant d'effectuer des recherches sur leur corpus, nous avons exploré nos différentes sources primaires.
Les deux journaux romands sont extrêmement fournis durant la période de la Première Guerre mondiale (1914-1918), avec 164'882 articles pour la Gazette de Lausanne et 56'126 pour le Journal de Genève:

\begin{itemize}
    \item 32'174 articles contenant le mot clé \og guerre \fg{}
    \item 282 articles contenant les mots clés \og Wille \fg{} et \og général \fg{}
    \item 320 articles contenant les mots clés \og Belgique \fg{} et \og invasion \fg{}
    \item 201 articles contenant les mots clés \og affaire \fg{} et \og colonels \fg{}
    \item 72 articles contenant les mots clés \og Hoffmann \fg{} et \og Grimm \fg{}
    \item 367 articles contenant le mot clé \og grève générale \fg{}
\end{itemize}

Cette exploration quantitative préliminaire nous confirme que l'on a un corpus contenant des données analysables afin de répondre à notre problématique du côté de la presse romande.

Par contre, les articles provenant de la NZZ sont beaucoup plus compliqués à analyser.
Ainsi, nous nous baserons également sur la \og Bote vom Untersee und Rhein \fg{} (TG), l' \og Oberländer Tagblatt \fg{} (BE), la \og Tagblatt der Stadt Thun \fg{} (BE), la \og Chronik der Stadt Zürich \fg{} (ZH) et le \og Zürchersische Freitagszeitung \fg{} (ZH), présents sur la plate-forme \textit{E-Newspaper Archives}.


\section*{Méthodologie}

Notre analyse se basera sur les articles parus durant la Grande Guerre.
Une étude comparative sera faite entre le corpus de la Neue Zürcher Zeitung ainsi que ceux de la Gazette de Lausanne et du Journal de Genève.
Celle-ci se basera notamment sur une analyse de la fréquence des articles mentionnant ces évènements et du champ lexical utilisé pour un même évènement.

Le but de cette analyse est de détecter quelle est l'amplitude du clivage linguistique entre les journaux, ce qui montrerait les différences d'opinion publique entre les Suisses.
Un des grands défis est de pouvoir comparer des champs lexicaux entre deux langues différentes.
En effet, il faudra trouver des liens de similarité entre champs lexicaux français et allemand.

Nous aimerions aussi analyser si ces clivages évoluent dans le temps.
La notion chronologique semble pertinente car la Première Guerre mondiale s'étend sur 4 ans et que les événements retenus jalonnent toute cette période.

Une limitation majeure provient du fait que le corpus de la NZZ sur Impresso n'est pas de la même qualité que pour les journaux suisses romands retenus.
Cela rendra l'analyse automatique très difficile.
En effet, il y a seulement 7044 articles de la NZZ recensés sur Impresso pour la période étudiée.
Après investigation, nous remarquons que les pages ne sont pas découpées en articles. De plus, la qualité de la reconnaissance de caractères est moins bonne que pour les deux journaux romands étudiés.
Cela vient du fait que la police d'écriture utilisée -- gothique -- est plus compliquée à reconnaître.

Dans le pire des cas, l'analyse du côté germanophone se fera manuellement.
Dans cette optique, les articles pourront être pris dans d'autres journaux suisses allemands.
Nous pourrions pour cela nous appuyer sur les sources recensées sur \url{https://www.e-newspaperarchives.ch/}.
À défaut d'avoir l'analyse se portant sur une masse de donnée importantes, nous avons quand même plusieurs points de vue germanophones.
Nous avons évalué les différentes archives disponibles sur le site sus-mentionné et avons retenu 5 journaux supplémentaires qui pourraient venir enrichir les articles de la \textit{Neue Züricher Zeitung} trouvés sur la plate-forme Impresso:

\begin{itemize}
    \item[---] [A] \textit{Bote vom Untersee und Rhein}
    \item[---] [B] \textit{Oberländer Tagblatt}
    \item[---] [C] \textit{Tagblatt der Stadt Thun}
    \item[---] [D] \textit{Chronik der Stadt Zürich}
    \item[---] [E] \textit{Zürcherische Freitagszeitung}
\end{itemize}

Ces journaux proviennent des régions de Berne, Thoune, Zurich et Thurgovie.
Le but étant de couvrir une plus grande région de la Suisse alémanique afin de mieux pouvoir interpréter le ressentiment général et ne pas se retrouver bloqué dans un journal trop biaisé.
Leurs orientations politiques sont variées: libéral-démocrate, conservateur, un journal de boulevard rapportant les faits divers et vies des célébrités locales, et un journal se voulant comme plate-forme neutre ou encore un autre plutôt bourgeois.

\begin{table}[h!]
\begin{tabular}{|c|c|c|c|c|c|c|}
\hline
 & Général Wille & Aff. des colonels &  Spitteler & Invasion Belgique &  Hoffmann & Grève générale \\
\hline\hline
 [A] & 52 & 9 & 0 & 44 & 13 & 9\\
\hline
 [B] & 282 & 48 & 37 & 292 & 46 & 28\\
\hline
 [C] & 96 & 35 & 28 & 0 & 44 & 37\\
\hline
 [D] & 47 & 0 & 31 & 20 & 3 & 6 \\
\hline
 [E] & 11 & 8 & 8 & 46 & 9 & 0 \\
\hline
\end{tabular}
\centering
\caption{
Statistiques sur le nombre d'articles dans les journaux alémaniques par mot-clé.
Les recherches ont été effectuées avec les mots-clés suivants sur la période 1910-1919 sauf mentionné autrement \textit{General Wille}, \textit{Obersten Affäre}, \textit{Spitteler} (en 1914-1915), \textit{Belgien} (en 1914), \textit{Fall Hoffmann, Fall Grimm, Hoffmann Grimm}, \textit{Landesstreik}.
}
\end{table}

\section*{Enjeux de la recherche}

Tout l'enjeu réside dans le fait d'analyser les dissensions entre les Suisses romands et alémaniques dans cette période clé de l'histoire.
De nos jours, les différences entre les régions linguistiques (Röstigraben, etc.) sont plutôt légères et humoristiques.
Il est cependant bon de se rappeler que nous n'en sommes pas arrivés là sans conflit.
Apprendre du passé afin d'éviter de revivre les précédentes tensions est nécessaire.

\begin{thebibliography}{9}

\subsection*{Sources}

\bibitem{krieg}
\bsc{Freymond}, Jacques et al. (ed.). Documents Diplomatiques Suisses, vol. 6, doc. 137: \og Le Général U. Wille au Chef du Département politique, A. Hoffmann \fg{} (20 juillet 1915). Berne, 1981. Repéré à \url{https://dodis.ch/43412}

\bibitem{whistle_blower}
\bsc{Freymond}, Jacques et al. (ed.). Documents Diplomatiques Suisses, vol. 6, doc. 160: \og A. Langie, cryptographe auprès de l'Etat-Major Général de l'Armée suisse au Chef du Département militaire, C. Decoppet \fg{} (8 décembre 1915). Berne, 1981. Repéré à \url{https://dodis.ch/43435}

\bibitem{verdict}
Après le verdict. \textit{Gazette de Lausanne}, p. 5. Lausanne, 1\ier{} mars 1916.

\bibitem{apaisement}
\bsc{Freymond}, Jacques et al. (ed.). Documents Diplomatiques Suisses, vol. 6, doc. 54: \og Aufruf an das Schweizervolk \fg{} (1\ier{} octobre 1914). Berne, 1981. Repéré à \url{https://dodis.ch/43329}

\bibitem{standpunkt}
\bsc{Spitteler}, Carl.
\textit{Unser Schweizer Standpunkt}. Zurich, 14 décembre 1914.

\bibitem{massacre}
Au jour le jour. \textit{Gazette de Lausanne}, p. 1. Lausanne, 21 Novembre 1914.

\subsection*{Littérature secondaire}

\bibitem{division}
\bsc{du Bois}, Pierre.
\textit{Union et division des Suisses. Les relations entre Alémaniques, Romands et Tessinois au XIXe et au XXe siècle}. Lausanne, 1983.

\bibitem{wahl}
\bsc{Sprecher}, Daniel.
Intrigen, Verzögerungen und ein abendlicher Canossagang. \textit{Neue Zürcher Zeitung}. Zurich, 2 août 2014.
Repéré à \url{https://www.nzz.ch/schweiz/intrigen-verzoegerungen-und-ein-abendlicher-canossagang-1.18355089}

\bibitem{delire}
\bsc{Meienberg}, Nicolas.
\textit{Le Délire général. L'armée suisse sous influence.}. Edition Zoe, 1988.

\bibitem{rede_spitteler}
\bsc{Münger}, Felix.
Die Rede, für die Carl Spitteler bitter bezahlen musste. \textit{Schweizer Radio und Fernsehen}. Zurich, 4 avril 2019.
Repéré à \url{https://www.srf.ch/kultur/literatur/unser-schweizer-standpunkt-die-rede-fuer-die-carl-spitteler-bitter-bezahlen-musste}

\bibitem{propagande}
\bsc{Stephens}, Thomas.
Une Suisse inondée de propagande.
\textit{Swissinfo}. Berne, 2 septembre 2014.
\url{https://www.swissinfo.ch/fre/culture/premi\%C3\%A8re-guerre-mondiale_une-suisse-inond\%C3\%A9e-de-propagande/40585506}

\bibitem{place}
\bsc{Elsig}, Alexandre.
\textit{Un ”laboratoire de choix”? La place de la Suisse dans le dispositif européen de la propagande allemande (1914-1918)}.
Histoire de la Suisse et des Suisses, Editions Payot, 1982.

\bibitem{sprachenfrieden}
\bsc{Büchi}, Christophe.
Der holprige Weg zum schweizerischen Sprachenfrieden. \textit{Neue Zürcher Zeitung}. Zurich, 18 novembre 2018.
Repéré à \url{https://www.nzz.ch/schweiz/der-holprige-weg-zum-schweizerischen-sprachenfrieden-ld.1437249}

\bibitem{HistoireJost}
\bsc{Jost}, Hans Ulrich.
\textit{Menace et Repliement}.
Histoire de la Suisse et des Suisses, Editions Payot, 1982.

\bibitem{HistoireWalter}
\bsc{Walter}, François.
\textit{Histoire de la Suisse. La création de la Suisse moderne (1830-1930)}. Neuchâtel, Alphil-Presses universitaires suisses, 2013.


\bibitem{ilot}
\bsc{Brand Crémieux}, Marie-Noëlle.
\textit{1914-1918 : la Suisse, un îlot dans la tourmente ?}. Université Nice, 2017.

\end{thebibliography}

\end{document}

