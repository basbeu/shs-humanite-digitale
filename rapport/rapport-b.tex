\documentclass[french,a4paper]{article}

\usepackage[utf8]{inputenc}          % Encodage des entrées
\usepackage[T1]{fontenc}             % Encodage de la fonte
\usepackage{babel}                   % Paquet de langue pour le français
\usepackage[margin=2.5cm]{geometry}  % Définition des marges
\usepackage{enumitem}                % Modification esthétique des listes

\PassOptionsToPackage{hyphens}{url}  % Les URL peuvent être écrits sur plusieurs lignes
\usepackage{hyperref}                % Les URL sont cliquables

\setlength{\parindent}{2em}          % Espace horizontal au début des paragraphes
\setlength{\parskip}{0.5em}          % Espace vertical entre les paragraphes
\renewcommand{\baselinestretch}{1.1} % Espace vertical entre les lignes

\title{La Première Guerre mondiale à travers les tensions entre les Suisses romands et alémaniques}
\author{\bsc{Beuchat} Bastien \and \bsc{Ding} Markus \and \bsc{Jollès} Eric \and \bsc{Mamie} Robin}
\date{6 décembre 2019} % Date du second rendu intermédiaire

%\date{25 mars 2020} % Date du premier rendu intermédiaire avec méthodologies et interprétations
%\date{22 avril 2020} % Date du second rendu intermédiaire avec méthodologies et interprétations
%\date{6 mai 2020} % Date du rendu final

\begin{document}

\maketitle

\section*{Introduction}

\subsection*{Contexte historique}

En 1914, l’Europe est sous haute tension.
L’assassinat à Sarajevo de l'héritier du trône austro-hongrois, l'archiduc François-Ferdinand, est l’étincelle qui déclenche la Première Guerre mondiale.
La Suisse se retrouve alors au centre du conflit, encerclée de toutes parts par les nations belligérantes qui constituent la Triple-Entente (Allemagne, Autriche-Hongrie, Italie) et la Triple-Alliance (France, Royaume-Uni, Russie).

La Suisse romande et la Suisse alémanique, de part leurs proximités culturelles et linguistiques avec les belligérants, ont différentes visions du conflit \cite{division}; l'unité du pays semble compromise.
L'élection du général de l'armée suisse Ulrich Wille \cite{wahl} va définitivement rompre celle-ci.
En effet plusieurs aspects le desservent: né à Hambourg, il est d'origine allemande et parle le Hochdeutsch à la place d'un dialecte suisse allemand.
Les Suisses romands dénoncent sa proximité culturelle, politique et militaire avec l'Allemagne pendant la guerre; il proposa par exemple d'entrer en guerre au côté des Empires centraux \cite{krieg}.

La plupart des journaux des deux régions linguistiques attisent alors les tensions en reprenant la propagande française ou allemande \cite{place} \cite{propagande} comme par exemple pour l'invasion de la Belgique par l'Allemagne. D'un côté la presse romande dénonce la violation de la neutralité belge et la cruauté allemande \cite{massacre}, alors que de l'autre, dans la presse alémanique, l'indifférence prévaut.

Des appels à l’apaisement sont lancés pour calmer les tensions croissantes; les plus importants sont celui de la Confédération \cite{apaisement} et celui de l'écrivain, essayiste et poète suisse Carl Spitteler, futur prix Nobel de littérature, avec son discours \og \textit{Unser Schweizer Standpunkt} \fg{} (\og \textit{Notre Point de vue suisse} \fg{}) \cite{standpunkt} \cite{rede_spitteler}.

Par ailleurs, depuis le début de la guerre, des colonels de l'armée suisse transmettent les bulletins de l'état-major et des dépêches diplomatiques décryptées à l'Allemagne et à l'Autriche.
Ces dépêches contiennent des informations sur les intentions militaires de la Triple-Entente, confirmant le penchant germanophile de l'état-major suisse \cite{whistle_blower}.
Quand à la fin de l'année 1915, cette affaire, nommée affaire des colonels, éclate au grand jour, la Suisse romande s'indigne de la gravité des actes commis et de la légèreté des sanctions prises contre les principaux responsables \cite{verdict}.

Les Suisses romands se sentent alors mis à l'écart dans les prises de décision de l'exécutif \cite{exclusion}, en effet un seul des conseillers fédéraux est alors romand.
Cela se confirmera au printemps 1917 lorsque le conseiller fédéral Arthur Hoffmann et le socialiste suisse Robert Grimm tenteront de négocier une paix séparée entre la Russie et l'Allemagne. \cite{hoffmann}
La découverte de cette intervention secrète crée de grands remous en Romandie ainsi que des pressions de la part de l'Entente qui craignent un retrait russe.
Les actions de Hoffmann, contraires aux principes de neutralité, l'ont forcé à démissionner.
Pour apaiser la situation, l'Assemblée fédérale le remplace par le socialiste genevois Gustave Ador, plutôt favorable à l'Entente.

À la fin de la guerre, un comité est formé à Olten afin de coordonner une grève générale dans tout le pays.
L'impact de ce mouvement est réduit par les tensions installées depuis le début de la guerre.
En effet, les Romands sont suspicieux de ce mouvement, composé uniquement de Suisses allemands; ils l'accusent de saboter la joie de la victoire de l'Entente en déclarant la grève le jour de l'armistice.
La grève est donc moins suivie et respectée en Suisse romande \cite{sprachenfrieden}.

La Suisse, loin de son image de pays paisible et uni, est pendant la Grande Guerre proche de l'implosion entre les différentes zones linguistiques.
Pour analyser les dissensions entre les Suisses romands et alémaniques lors de cette période clé de l'histoire, nous allons étudier les évènements cités tels qu'ils apparaissent dans les presses des deux régions linguistiques.


\section*{Corpus et références bibliographiques}

Nous nous attacherons à comparer les divergences de points de vue entre Suisses romands et allemands durant la Première Guerre mondiale.
Dans cette optique, nous travaillons avec des archives des presses romande et alémanique.
Nous étudions donc les articles parus dans la Gazette de Lausanne et le Journal de Genève pour analyser le point de vue francophone et ceux de la Neue Zürcher Zeitung (NZZ) pour l’analyse germanophone.
Ces trois journaux ont une ligne éditoriale libérale \cite{clavien} à cette époque et se vendent également en France et en Allemagne.
Nous travaillons principalement sur des articles parus pendant les moments critiques décrits dans la première partie.

L'interface Impresso permettant d'effectuer des recherches sur leur corpus, nous avons exploré nos différentes sources primaires.
Les deux journaux romands sont extrêmement fournis durant la période de la Première Guerre mondiale (1914-1918), avec 164'882 articles pour la Gazette de Lausanne et 56'126 pour le Journal de Genève:

\begin{itemize}
    \item 32'174 articles contenant le mot clé \og guerre \fg{}
    \item 282 articles contenant les mots clés \og Wille \fg{} et \og général \fg{}
    \item 320 articles contenant les mots clés \og Belgique \fg{} et \og invasion \fg{}
    \item 201 articles contenant les mots clés \og affaire \fg{} et \og colonels \fg{}
    \item 72 articles contenant les mots clés \og Hoffmann \fg{} et \og Grimm \fg{}
    \item 367 articles contenant le mot clé \og grève générale \fg{}
\end{itemize}

Cette exploration quantitative préliminaire nous confirme que nous avons un corpus contenant des données analysables afin de répondre à notre problématique du côté de la presse romande.
Néanmoins, nous nous apercevons que certaines données sont faibles, par exemple pour l'affaire Hoffmann.
D'autres mots clés apporteront des informations complémentaires.

En outre, une limitation majeure provient du fait que le corpus de la NZZ sur Impresso n'est pas de la même qualité que pour les journaux suisses romands retenus.
De plus, il y a seulement 7'044 articles recensés sur cette période.
Cela rendra l'analyse automatique très difficile.
Ainsi, selon la qualité des données, nous pourrions également nous baser sur d'autres journaux germanophones présents sur la plate-forme \textit{E-newspaper}, comme détaillés dans la méthodologie.


\section*{Méthodologie}

Le but de notre analyse est de détecter quelle est l'amplitude du clivage linguistique entre les journaux, ce qui montrerait les différences d'opinion publique entre les Suisses.
Un des grands défis est de pouvoir comparer des champs lexicaux entre deux langues différentes; notre étude se basera notamment sur une analyse de la fréquence des articles mentionnant ces évènements et donc leur champs lexicaux.
En effet, il faudra trouver des similarités -- ou des dissemblances -- entre les résultats du côté francophone et germanophone.

Nous aimerions aussi analyser si ces clivages évoluent dans le temps.
La notion chronologique semble pertinente car la Première Guerre mondiale s'étend sur 4 ans et que les événements retenus jalonnent toute cette période.

Après investigation, nous avons observé des problèmes avec le corpus de la NZZ.
Nous remarquons que les pages du journal ne sont pas découpées en articles.
De plus, la qualité de la reconnaissance de caractères gothique est moins bonne que pour les deux journaux romands étudiés, causant un certain défi technique pour son analyse.
Ceci produit du bruit dans la transcription, mais nous pourrions nous baser sur un indice de confiance lors de notre analyse.

Si cette analyse s'avère trop difficile, elle se fera manuellement du côté germanophone.
La comparaison sera donc plus difficile avec la presse romande, car elle offre une analyse plus qualitative que quantitative.
Mais elle offrira tout de même un bon aperçu de l'opinion générale tout en prenant des précautions dans la conclusion.

Dans cette optique, les articles pourront être pris dans d'autres journaux suisses allemands.
Nous pourrions pour cela nous appuyer sur les sources recensées sur \url{https://www.e-newspaperarchives.ch/}.
À défaut d'analyser une masse de donnée importante, car nous n'avons pas accès à une API pour télécharger les données depuis cette plate-forme, nous avons quand même plusieurs points de vue germanophones.
Voici les détails des autres journaux germanophones, retenus au cas où la NZZ ne nous conviendrait pas totalement:

\begin{itemize}
    \item[---] [A] \textit{Bote vom Untersee und Rhein}: se considère comme une plate-forme neutre et le porte-parole de la région de Thurgovie
    \item[---] [B] \textit{Oberländer Tagblatt}: journal progressiste-bourgeois et libéral-démocrate de l'Oberland bernois
    \item[---] [C] \textit{Tagblatt der Stadt Thun}: journal bernois d'orientation libérale-démocratique
    \item[---] [D] \textit{Chronik der Stadt Zürich}: journal de boulevard couvrant les célébrités et faits divers locaux de la ville de Zurich
    \item[---] [E] \textit{Zürcherische Freitagszeitung}: presse conservatrice et proche du parti \textit{FDP} (libéral-démocrate) de la région zurichoise
\end{itemize}

La décision d'utiliser ou non un article se fera au cas par cas en fonction de la qualité de ce dernier et du sujet abordé.

\begin{table}[h!]
\begin{tabular}{|c|c|c|c|c|c|c|}
\hline
 & Général Wille & Aff. des colonels &  Spitteler & Invasion Belgique &  Hoffmann & Grève générale \\
\hline\hline
 [A] &  52 &  9 &  0 &  44 & 13 &  9 \\
\hline
 [B] & 282 & 48 & 37 & 292 & 46 & 28 \\
\hline
 [C] &  96 & 35 & 28 &   0 & 44 & 37 \\
\hline
 [D] &  47 &  0 & 31 &  20 &  3 &  6 \\
\hline
 [E] &  11 &  8 &  8 &  46 &  9 &  0 \\
\hline
\end{tabular}
\centering
\caption{
Statistiques sur le nombre d'articles dans les journaux alémaniques par mot-clé.
Les recherches ont été effectuées avec les mots-clés suivants sur la période 1910-1919 sauf mentionné autrement \textit{General Wille}, \textit{Obersten Affäre}, \textit{Spitteler} (en 1914-1915), \textit{Belgien} (en 1914), \textit{Fall Hoffmann, Fall Grimm, Hoffmann Grimm}, \textit{Landesstreik}.
}
\end{table}


\begin{thebibliography}{9}

\subsection*{Sources}

\bibitem{krieg}
\bsc{Freymond}, Jacques et al. (ed.).
Documents Diplomatiques Suisses, vol. 6, doc. 137: \og Le Général U. Wille au Chef du Département politique, A. Hoffmann \fg{} (20 juillet 1915).
Berne, 1981. Repéré à \url{https://dodis.ch/43412}

\bibitem{massacre}
Auteur inconnu.
Au jour le jour. \textit{Gazette de Lausanne}, p. 1.
Lausanne, 21 Novembre 1914.

\bibitem{apaisement}
\bsc{Freymond}, Jacques et al. (ed.).
Documents Diplomatiques Suisses, vol. 6, doc. 54: \og Aufruf an das Schweizervolk \fg{} (1\ier{} octobre 1914).
Berne, 1981.
Repéré à \url{https://dodis.ch/43329}

\bibitem{standpunkt}
\bsc{Spitteler}, Carl.
\textit{Unser Schweizer Standpunkt}.
Zurich, 14 décembre 1914.

\bibitem{whistle_blower}
\bsc{Freymond}, Jacques et al. (ed.).
Documents Diplomatiques Suisses, vol. 6, doc. 160: \og A. Langie, cryptographe auprès de l’Etat-Major Général de l’Armée suisse au Chef du Département militaire, C. Decoppet \fg{} (8 décembre 1915).
Berne, 1981. Repéré à \url{https://dodis.ch/43435}

\bibitem{verdict}
Auteur inconnu.
Après le verdict. \textit{Gazette de Lausanne}, p. 5.
Lausanne, 1\ier{} mars 1916.

\bibitem{exclusion}
\bsc{Freymond}, Jacques et al. (ed.).
Documents Diplomatiques Suisses, vol. 6, doc. 214: \og Lettre collective des Cantons romands concernant les négociations avec l’Allemagne \fg{} (5 octobre 1916).
Berne, 1981. Repéré à \url{https://dodis.ch/43489}

\subsection*{Littérature secondaire}

\bibitem{division}
\bsc{du Bois}, Pierre.
\textit{Union et division des Suisses. Les relations entre Alémaniques, Romands et Tessinois au XIXe et au XXe siècle}.
Lausanne, 1983.

\bibitem{place}
\bsc{Elsig}, Alexandre.
\textit{Un ”laboratoire de choix”? La place de la Suisse dans le dispositif européen de la propagande allemande (1914-1918)}.
Histoire de la Suisse et des Suisses, Editions Payot, 1982.

\bibitem{hoffmann}
\bsc{Stauffer}, Paul.
\textit{Die Affäre Hoffmann-Grimm}.
Schweizer Monatshefte, p. 1-30, 1973-1974.

\bibitem{clavien}
\bsc{Clavien}, Alain.
\textit{Grandeurs et misères de la presse politique}.
Editions Antipodes, 2010.

\bibitem{delire}
\bsc{Meienberg}, Nicolas.
\textit{Le Délire général. L’armée suisse sous influence}.
\bsc{Piccard}, Monique (trad.)
Edition Zoe, 1988 [1987].

\bibitem{grande_guerre}
\bsc{Rossfeld}, Roman \textit{et alii}.
\textit{14/18 La Suisse et la Grande Guerre}.
Editions Verlag Hier+Jetzt, 2014.

\bibitem{HistoireJost}
\bsc{Jost}, Hans Ulrich.
\textit{Menace et Repliement}.
Histoire de la Suisse et des Suisses, Editions Payot, 1982.

\bibitem{HistoireWalter}
\bsc{Walter}, François.
\textit{Histoire de la Suisse. La création de la Suisse moderne (1830-1930)}.
Neuchâtel, Alphil-Presses universitaires suisses, 2013.

\bibitem{ilot}
\bsc{Brand Crémieux}, Marie-Noëlle.
\textit{1914-1918 : la Suisse, un îlot dans la tourmente ?}.
Université Nice, 2017.

\subsection*{Médias}

\bibitem{wahl}
\bsc{Sprecher}, Daniel.
Intrigen, Verzögerungen und ein abendlicher Canossagang. \textit{Neue Zürcher Zeitung}.
Zurich, 2 août 2014.
Repéré à \url{https://www.nzz.ch/schweiz/intrigen-verzoegerungen-und-ein-abendlicher-canossagang-1.18355089}

\bibitem{propagande}
\bsc{Stephens}, Thomas.
Une Suisse inondée de propagande.
\textit{Swissinfo}. Berne, 2 septembre 2014.
\url{https://www.swissinfo.ch/fre/culture/premi\%C3\%A8re-guerre-mondiale_une-suisse-inond\%C3\%A9e-de-propagande/40585506}

\bibitem{rede_spitteler}
\bsc{Münger}, Felix.
Die Rede, für die Carl Spitteler bitter bezahlen musste. \textit{Schweizer Radio und Fernsehen}.
Zurich, 4 avril 2019.
Repéré à \url{https://www.srf.ch/kultur/literatur/unser-schweizer-standpunkt-die-rede-fuer-die-carl-spitteler-bitter-bezahlen-musste}

\bibitem{sprachenfrieden}
\bsc{Büchi}, Christophe.
Der holprige Weg zum schweizerischen Sprachenfrieden. \textit{Neue Zürcher Zeitung}.
Zurich, 18 novembre 2018.
Repéré à \url{https://www.nzz.ch/schweiz/der-holprige-weg-zum-schweizerischen-sprachenfrieden-ld.1437249}

\end{thebibliography}

\end{document}

